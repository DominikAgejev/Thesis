\documentclass[zavrsnirad]{fer}
% Dodaj opciju upload za generiranje konačne verzije koja se učitava na FERWeb
% Add the option upload to generate the final version which is uploaded to FERWeb


\usepackage{blindtext}
\usepackage{xcolor}
\usepackage{listings}

\lstset{ %
	language=Python,
	basicstyle=\ttfamily\footnotesize,
	keywordstyle=\color{blue},
	commentstyle=\color{green},
	stringstyle=\color{red},
	showstringspaces=false,
	frame=single,
	numbers=left,
	numberstyle=\tiny\color{gray},
	breaklines=true,
	breakatwhitespace=true,
	tabsize=2
}

\usepackage[labelfont=bf]{caption}

\usepackage{changepage} % For adjusting margins
\usepackage{multirow}

%--- PODACI O RADU / THESIS INFORMATION ----------------------------------------

% Naslov na engleskom jeziku / Title in English
\title{Optical character recognition system for older books in Croatian}

% Naslov na hrvatskom jeziku / Title in Croatian
\naslov{SUSTAV ZA OPTIČKO RASPOZNAVANJE TEKSTA STARIJIH KNJIGA NA HRVATSKOME JEZIKU}

% Broj rada / Thesis number
\brojrada{1605}

% Autor / Author
\author{Dominik Agejev}

% Mentor 
\mentor{Prof.\@ Tomislav Hrkać}

% Datum rada na engleskom jeziku / Date in English
\date{September, 2024}

% Datum rada na hrvatskom jeziku / Date in Croatian
\datum{rujan, 2024.}

%-------------------------------------------------------------------------------


\begin{document}


% Naslovnica se automatski generira / Titlepage is automatically generated
\maketitle


%--- ZADATAK / THESIS ASSIGNMENT -----------------------------------------------

% Zadatak se ubacuje iz vanjske datoteke / Thesis assignment is included from external file
% Upiši ime PDF datoteke preuzete s FERWeb-a / Enter the filename of the PDF downloaded from FERWeb
\zadatak{hr_0036537505_73.pdf}


%--- ZAHVALE / ACKNOWLEDGMENT --------------------------------------------------

\begin{zahvale}
  % Ovdje upišite zahvale / Write in the acknowledgment
  Jakovu, Lovri i Roku,
  hvala na pomoći!
\end{zahvale}


% Odovud započinje numeriranje stranica / Page numbering starts from here
\mainmatter


% Sadržaj se automatski generira / Table of contents is automatically generated
\tableofcontents


%--- UVOD / INTRODUCTION -------------------------------------------------------
\chapter{Uvod}
\label{pog:uvod}

Cilj rada nadići je uspješnost gotovih sustava za optičko raspoznavanje teksta (eng. \textit{Optical Character Recognition} ili \textit{OCR}) na starijim knjigama hrvatskoga jezika koristeći se nenadziranim metodama učenja, odnosno bez označenih podataka za trening modela, uz predobradu i naknadnu obradu.

Iako suvremeni sustavi poput DTrOCR-a \cite{Fujitake2023} postižu gotovo savršene rezultate u raznim primjenama, optičko raspoznavanje teksta  nipošto nije riješen problem. Još uvijek i najbolji sustavi, poput gore navedenog, pogrešno prepoznaju više od 10\% riječi na fotografijama teksta „u divljini“ i gotovo 20\% riječi u rukopisima na kineskom jeziku.

Uz to, zbog ovisnosti o jezičnim modelima, novija rješenja općenito nisu primjenjiva bez dodatne prilagodbe na manje zastupljene jezike, poput hrvatskog, ili se pak oslanjaju na veliku količinu označenih podataka ili na sintetičke podatke, generirane modelima koji za rjeđe jezike nisu dostupni, te zahtijevaju značajne računalne resurse.

Nadalje, zbog suviše uske primjene, specifični problemi, poput predmeta ovog rada, raspoznavanja teksta antikvarnih knjiga i to na jeziku ograničene uporabe, redovito se zaobilaze u prilog doprinosima aktualnim primjenama. \cite{Olejniczak2022}

U okviru ovog rada najprije se uvodi u područje, metode i temeljne pojmove koji će se koristiti u radu. Zatim je detaljno izložen zadatak i njegove specifičnosti uz pregled dosadašnjih postignuća unutar područja.

Prelazeći na izvedbu rješenja, razmatraju se najznačajniji slobodno dostupni OCR alati prikladni zadatku, a to su Tesseract \cite{Smith2007}, OCR sustav opće namjene koji održava Google, te Ocular \cite{Berg-Kirkpatrick2013}, razvijen specifično za primjenu na antikvarnim dokumentima.

Nakon treniranja i optimiziranja hiperparametara Oculara, uspoređen je s Tesseractom na već predobrađenim ispitnim podacima gdje se pokazuje da usprkos starijoj arhitekturi u bitnome nadjačava Tesseract, ali uz određena ograničenja.

Konačno, izveden je sustav glasanja kojim se postiže veća uspješnost od one samostalnih modela.


%	\begin{figure}[htb]
%	  \centering
%	  \includegraphics[width=0.38\linewidth]{Figures/lenovo_yoga_tab3_pro_front.png} 
%	  \caption{Moja prva slika}
%	  \label{slk:prvaslika}
%	\end{figure}

% Referenciramo se na sliku \ref{slk:prvaslika} u sredini rečenice, zatim prije zareza \ref{slk:prvaslika}, te zatim na kraju rečenice \ref{slk:prvaslika}.
% Upravo smo testirali radi li naredba \verb|\ref| ispravno u slučaju kada nakon nje slijedi točka.


%-------------------------------------------------------------------------------
\chapter{Uvod u OCR}
\label{pog:ocr_uvod}

Optičko raspoznavanje teksta grana je računalnog vida koja se bavi izdvajanjem teksta iz slika, bilo dokumenata, rukopisa ili scenskih fotografija, radi lakog pretraživanja i uređivanja, jednostavnijeg arhiviranja ili pak dostupnosti sadržaja slabovidnima i slijepima.

U odnosu na sadržaj ulaznih slika najčešće govorimo o prepoznavanju teksta tiskanih dokumenata, rukopisa ili teksta „u divljini“, npr. natpisa na pročeljima trgovina, s tim da je potonje uže povezano s drugim granama računalnog vida poput detekcije i klasifikacije objekata.

OCR sustavi često se razvijaju i za još uže definirane zadatke, primjerice prepoznavanje teksta na računima ili antikvarnim dokumentima. Takvi sustavi, kakvima se bavi i ovaj rad, nazivaju se jednonamjenskima (eng. \textit{task-specific}), dok se sustavi prilagođeni raznim uporabama zovu sustavima opće namjene (eng. \textit{general purpose}). \cite{Borovikov2014}

Optičko raspoznavanje teksta podrazumijeva u bitnome pet koraka: predobradu, segmentaciju, izdvajanje značajki, klasifikaciju te naknadnu obradu. \cite{Dhande2017} U ovom poglavlju izložit će se ugrubo najznačajnije metode i pojmovi koji će se koristiti u ostatku rada.

\section{Predobrada}

Predobrada se odnosi na postupak prilagodbe ulazne slike radi uspješnijeg raspoznavanja znakova. Načela po kojima se ravna predobrada uključuju pojednostavljenje ulaza izostavljanjem nebitnih informacija, što čine binarizacija i eliminacija šuma, te ispravljanje fizičkih nesavršenosti, uzrokovanih bilo tiskom bilo digitalizacijom ulaza, što rade metode poput ispravljanja nagnuća (eng. \textit{skew correction}).

\subsection{Binarizacija}
\label{Binarizacija}

Cilj binarizacije razlučivanje je između teksta i pozadine. Najjednostavniji način za to postavljanje je praga (eng. \textit{thresholding}) za koji su svi pikseli s RGB ili sivotonskim (eng. \textit{grayscale}) vrijednostima nižim od praga obojani crno, tj. označeni kao tekst, a pikseli iznad praga označeni kao pozadina. \cite{Jyotsna2016}

Razlikujemo globalne i lokalne metode binarizacije. Globalne, poput često korištene Otsuove  metode \cite{Otsu1979}, postavljaju jedan prag za cijeli dokument, što je vremenski učinkovito i uspješno u idealnom slučaju s jednoličnom pozadinom, no zakazuje pri nejednakom osvjetljenju ili sjeni uslijed loše skeniranog pregiba knjige. Lokalne, poput metode adaptivnog kontrasta \cite{Su2013}, temeljenoj na prepoznavanju rubova pomoću kontrasta susjednih piksela, nešto su resursno zahtjevnije, ali zato daju bolje rezultate. \cite{Otsu1979, Su2013}

\subsection{Ispravljanje nagnuća}

Cilj ispravljanja nagnuća zaokrenuti je retke teksta tako da su vodoravni. Utvrđivanje kuta nagnuća redaka binarizirane slike može se svesti na pronalazak pravca koji najbolje aproksimira redak, a u tu svrhu najčešće se koristi Houghova transformacija. \cite{Hassanein2015} Konceptualno, Houghova transformacija za svaku rubnu točku slike pronalazi parametre (m, c) pravaca koji se kroz nju mogu provući. Pronađeni skup parametara zapravo je pravac u m, c prostoru, a odredivši pripadajući pravac svakoj točci, ako neki pravci imaju zajedničko sjecište, kroz njima pripadne točke moguće je povući pravac koji u konkretnom slučaju određuje redak teksta.

\section{Segmentacija}

\section{Izdvajanje značajki}

\textcolor{red}{\textit{Neuralne mreže se koriste i za izdvajanje značajki, što sad s organizacijom poglavlja? \newline prof. Hrkać: Ništa, možete samo spomenuti da su starije metode koristile ručno definirane značajke (npr. tzv. Granlundove koeficijente) dok se kod današnjih metoda značajke uče, pa se izdvajanje značajki obavlja u sklopu iste neuronske mreže koja se koristi i za klasifikaciju (raniji slojevi izdvajaju (naučene) značajke, dok završni sloj provodi klasifikaciju). Dakle bit će vrlo kratko podpoglavlje :).}}

\section{Klasifikacijske arhitekture}

\subsection{Usporedba predložaka}

\subsection{CNN}

\subsection{RNN}

\subsection{LSTM}

\textit{long short-term memory} (\textit{LSTM})

\section{Naknadna obrada}

\textcolor{red}{\textit{Tesseract zamjena ',,' (dva zareza) s " (ravni navodnici)}}

%-------------------------------------------------------------------------------
\chapter{OCR starijih tekstova}
\label{pog:ocr_starijih_tekstova}

Iz specifičnosti zadatka, tj. starosti knjiga,  proizlaze određene poteškoće, naime:
\begin{itemize}
	\item Čest višak ili manjak tinte pri tisku pojedinih znakova
	\item \textcolor{red}{Zastarjela znakovlja (fontovi) s neuobičajenim znakovima} %(npr. ſ , ʒ )
	\item Otežana predobrada zbog spremanja na mikrofilmu
	\item Arhaičan jezik
	\item Neravan tisak
	\item Istrošenost i oštećenja papira
\end{itemize}



\section{Pregled literature}

\cite{Berg-Kirkpatrick2013}
\cite{Springmann2014}
\cite{Christy2017}
\cite{Wick2018}
\cite{Garrette2015}
\cite{Garrette2016}


%-------------------------------------------------------------------------------
\chapter{Tesseract}
\label{pog:tesseract}

Tesseract \cite{Smith2007} je najpoznatiji i najprecizniji slobodno dostupan OCR sustav opće namjene koji podržava i hrvatski jezik, a od 4. inačice temeljen je na LSTM neuronskim mrežama. U ovom poglavlju objasnit će se ugrubo Tesseractov proces prepoznavanja teksta prema zadanim postavkama.

\section{Predobrada}

Tesseract ima ugrađena tri koraka predobrade: binarizaciju, eliminaciju šuma i analizu uređenja stranica.

\subsection{Binarizacija}

Tesseract se koristi Otsuovom metodom, no ne za postavljanje jednog globalnog praga za čitavu stranicu, već rabi implementaciju Leptonica biblioteke \cite{Leptonica} koja dijeli stranicu u jednake blokove te na njima postavlja prag. Takvim pristupom nadvladavaju se varijacije u svjetlini na makro razini slike, ali se zadržava i veća resursna učinkovitost uslijed paralelizacije i izbjegavanja složenijih računa lokalnih metoda.

Za slike koje nisu više-manje dvobojne nego pate od većih nejednakosti u osvijetljenju Tesseract podržava i Sauvolinu \cite{Sauvola1997} lokalnu metodu binarizacije koja utvrđuje prag za svaki piksel slike na temelju srednje vrijednosti i standardne devijacije okolnih piksela.

\subsection{Eliminacija šuma}

Razlučivši pozadinu od ostatka slijedi brisanje šuma, tj. smetnji poput prolivene tinte. To se postiže pronalaskom spojenih piksela te usporedbom karakteristika nakupine s tipičnim karakteristikama teksta.

Tijekom analize razmatra se: \cite{Tesseract}

\begin{description}
	\item[Širina poteza] \hfill \\ Potezi jednolične širine vjerojatnije pripadaju znaku.
	\item[Veličina nakupina] \hfill \\ Skupine piksela koje se protežu izvan uobičajene visine retka vjerojatno nisu znakovi.
	\item[Obujmljena površina] \hfill \\ Gledajući površinu koju skupina piksela okružuje možemo procijeniti je li znak ili nije.
	\item[Broj nakupina po retku] \hfill \\ Brojeći skupine piksela u retku provjerava se omjer malih nakupina naspram skupina veličine znaka.
 	\item[Odnos među točkama] \hfill \\ Ako se detektira velik broj susjednih točaka na istoj visini ne odbacuju se već su označene kao "vodeće točke" sadržaja.
\end{description}

\subsection{Analiza uređenja stranice}

\begin{description}
	\item[Line Detection] \hfill Tesseract uses Leptonica to find and remove rule/separator lines in the input image. This helps to separate text from graphical elements like tables or forms.
	\item[Photo Region Detection] \hfill The \texttt{FindImages} function from Leptonica is used to detect photo regions in the input. This allows Tesseract to distinguish between text and image areas.
	\item[Connected Component Analysis] \hfill This step identifies individual characters or character fragments. It is performed by the \texttt{find\_components} function, which scans the binary image pixel by pixel, labels connected black pixels, and groups them into connected components representing potential characters or parts of characters.
	\item[Orientation and Script Detection] \hfill If enabled, Tesseract analyzes the layout to detect text orientation and script. This is particularly useful for documents that may contain text in multiple orientations or scripts.
	\item[Column Detection] \hfill If \texttt{PSM\_COL\_FIND\_ENABLED} is true for the selected page segmentation mode, Tesseract attempts to divide the image into columns. This is crucial for correctly processing multi-column documents.
	\item[Text Line Formation] \hfill Tesseract analyzes the spatial relationships between connected components to detect text lines and words. This step uses statistical approaches based on the spacing between components.
\end{description}

\cite{Breuel2013} o LSTM-u


%-------------------------------------------------------------------------------
\chapter{Ocular}
\label{pog:ocular}

Ocular \cite{Berg2013} je sustav za optičko raspoznavanje teksta razvijena specifično za rad s povijesnim dokumentima, i koji je, kada je izdan i svojevremeno unaprjeđen \cite{Berg2014}, bio vrhunac tehnologije za to područje (eng. \textit{state-of-the-art}).

Njegove glavne značajke su: \cite{Ocular}

\begin{itemize}
	\item Nenadzirano učenje nepoznatih znakovlja rabeći slike ulaznog dokumenta i korpus teksta na ciljnom jeziku.
	\item Prilagođenost radu sa šumovitim dokumentima.
	\item Podrška za višejezične dokumente.
	\item Nenadzirano učenje ortografskih varijacija uslijed arhaičnog pravopisa.
	\item Istovremen ispis doslovnog teksta i normaliziranog oblika (prilagođenog standardnom jezku).
\end{itemize}

%-------------------------------------------------------------------------------
\chapter{Metodologija}
\label{pog:metodologija}

Ključan dio razvoja boljeg rješenja evaluacija je preciznosti Tesseracta i Oculara. U tu svrhu potreban je ispitni skup podataka prilagođen ograničenjima sustava. 

Zatim treba prikupiti podatke za treniranje Oculara i namjestiti njegove hiperparametre da daju zadovoljavajuće rezultate, što će zapravo biti najznačajniji dio rada.


\section{Ispitni skup podataka}

Budući da Tesseract ima ugrađenu predobradu slika, radi pravednije usporedbe same klasifikacije teksta izabrani su već obrađeni dokumenti.

Izvadci iz:
\begin{itemize}
	\item Fra Jozo Garić, biskup – Korizmena okružnica (1932.)
	\textcolor{red}{\item Prof. dr. Antun Heinz – Nekoliko misli o definiciji i klasifikaciji plodova (1897.)
	\item Sv. Petar Kanizije – Summa nauka christianskoga (1583.)}
\end{itemize}

Oba sustava imaju određena ograničenja na ulaze: 

Ocular radi jedino s PDF dokumentima zastarjele verzije 1.4 te ih je stoga bilo potrebno pretvoriti u taj format. Za to je korišten Ghostscript, \cite{Ghostscript} slobodno dostupan alat otvorenog koda. 

Tesseract pak radi jedino na slikama te je zato bilo potrebno ekstrahirati ih iz PDF-a prije prepoznavanje teksta. Ovdje je zgodno napomenuti da treba paziti da se prilikom pretvorbe ne smanji DPI rezolucija jer to ima poguban utjecaj na preciznost.

\section{Mjere uspješnosti}

Za uspoređivanje znakovnih nizova najčešće korištena mjera uspješnosti je \textbf{Levenshteinova udaljenost} koja bilježi broj potrebnih zamjena, brisanja ili umetanja znakova da bi se iz jednog niza dobio drugi. \cite{Levenshtein1965}

Budući da ta mjera ovisi o duljini teksta, dijeljenjem Levenshteinove udaljenosti ukupnim brojem znakova dobivamo \textbf{stopu pogreške za znakove} (eng. \textit{Character Error Rate}). Obično se dobrom vrijednošću smatra 1-2\%.

\textbf{Stopa pogreške za riječi} (eng. \textit{Word Error Rate}) dobiva se uzimanjem riječi za najmanju jedinicu zamjene pri računanju Levenshteinove udaljenosti, tj. ako su jedan ili više znakova u riječi pogrešni čitava riječ broji se kao pogrešna, te dijeljenjem te udaljenosti s ukupnim brojem riječi.


\subsection{F1?}

\subsection{Recall?}


%-------------------------------------------------------------------------------
\chapter{Optimizacija Oculara}
\label{pog:optimizacija_oculara}

Za razliku od Tesseracta, Ocular nije univerzalno primjenjiv za različite jezike i znakovlja već je potrebno naučiti jezični model, za koji je potreban korpus teksta na ciljnom jeziku, te model fonta, koji izučava na temelju slika čiji će tekst kasnije prepoznavati.

\section{Jezični model}

Pri izgradnji skupa podataka za trening jezičnog modela najrelevantnije su dvije stavke: broj podataka i tematika teksta.

U izvornom radu pokazano je kako model malo precizniji (4 WER postotna boda) na dokumentima čija je tematika pokrivena u jezičnom modelu. Budući da je sustav namijenjen starijim knjigama, od kojih je dobar dio vjerske tematike, uključeno je Sveto Pismo i druge duhovne knjige pored novijeg i starijeg štiva koje doprinosi većoj raznolikost izričaja i opsežnijem rječniku.
\cite{Berg2013}

\textcolor{red}{Izvorno} trenirano na tekstovima javno dostupnih knjiga, poput djela Augusta Šenoe, Marije Jurić-Zagorke, Charlesa Dickensa i sl. (7.5 milijuna riječi) uz Šarićev prijevod Svetog Pisma (670k riječi) i još 11 knjiga vjerske tematike (500k riječi). 

Ukupan broj riječi od 8.7 milijuna usporediv je sa skupom podataka korištenim u izvornom radu koji ih ima 10 milijuna. 

\bgroup
\def\arraystretch{1.25}
\begin{table}[h]
	\centering
	\begin{tabular}{|c|c|c|c|c|}
		\hline
		\multirow{2}{*}{\textbf{Iteracije treninga}} & \multicolumn{2}{|c|}{\textbf{Veličina snopa}} & \multicolumn{1}{|c|}{\multirow{2}{*}{\textbf{CER}}} & \multicolumn{1}{|c|}{\multirow{2}{*}{\textbf{WER}}} \\ \cline{2-3}
		
		& \textbf{Trening}  & \textbf{Transkripcija}  & \multicolumn{1}{|c|}{}  & \multicolumn{1}{|c|}{}  \\ \hline
		3x3    & 10    & 10 &  2.03 & 3.58  \\ \hline
		3x3 & 50  & 50 & \textbf{1.06} & \textbf{2.76}   \\ \hline
		3x3+2x2 & 50 & 50 & 1.19 & 2.9 \\ \hline                                              
	\end{tabular}
	\caption{\textcolor{red}{Trebala bi ići u drugi odjeljak za trening fonta}}
	\label{tab:__________}
\end{table}
\egroup


\subsection{Proširenje podacima van domene}

\subsection{Proširenje nesavršenim podatcima}

\bgroup
\def\arraystretch{1.25}
\begin{table}[h]
	\centering
	\begin{tabular}{|c|c|c|c|c|}
		\hline 
		\multicolumn{2}{|c|}{\textbf{Veličina snopa}} & \multicolumn{1}{|c|}{\multirow{2}{*}{\textbf{CER}}} & \multicolumn{1}{|c|}{\multirow{2}{*}{\textbf{WER}}} \\ \cline{1-2}
		
		\textbf{Trening}  & \textbf{Transkripcija}  & \multicolumn{1}{|c|}{}  & \multicolumn{1}{|c|}{}  \\ \hline
		40 & 40 &  1.59 & 3.64   \\ \hline
		40 & 120 & 1.81 & 3.77   \\ \hline
		120 & 50 & 2.21 & 5.04 \\ \hline        
		120 & 120 & 2.29 & 5.78 \\ \hline                                             
	\end{tabular}
	\caption{\textcolor{red}{dd}}
	\label{tab:_________2}
\end{table}
\egroup

\section{Ispitivanje ortografskih značajki}


%-------------------------------------------------------------------------------
\chapter{Sinteza rješenja}
\label{pog:sinteza_rješenja}

Budući da je izuzev ispuštanja određenih redaka Ocular točniji od Tesseracta, kako je utvrđeno u prethodnom poglavlju, ovdje predlažemo jednostavan sustav glasanja kojim je Ocularova manjkavost otklonjena bez gubitka preciznosti.

Algoritam glasanja čita redak po redak Tesseractov ispis i traži odgovarajući redak Ocularovog ispisa na temelju sličnosti izračunate pomoću Levenshteinove udaljenosti. Ako pronađe dovoljno sličan redak odabire ga kao izlaz, inače preferira Tesseractov ispis.

\begin{figure}[h]
	\centering
	\begin{adjustwidth}{1cm}{1cm}
		\begin{lstlisting}
			for t_line in tesseract_output:
				for c_line in ocular_output:
					distance = Levenshtein.distance(t_line, c_line)
					if distance < threshold * len(t_line):
						output.append(c_line)
					break
				output.append(t_line)
		\end{lstlisting}
	\end{adjustwidth}
	\caption{Algoritam glasanja predstavljen Python kodom}
	\label{fig:python_code}
\end{figure}

\bgroup
\def\arraystretch{1.25}
\begin{table}[h]
	\centering
	\begin{tabular}{|c|c|c|c|}
		\hline
		\textbf{OCR sustav} & \textbf{Pojedinosti} & \textbf{CER} & \textbf{WER} \\ \hline
		Ocular & & 8.83 & 10.94 \\ \hline
		Ocular & Zanemareni retci s CER>20 & 1.06 & 2.76 \\ \hline
		Tesseract & & 1.52 & 3.77 \\ \hline
		Predložen sustav & & \textbf{0.96} & \textbf{2.48} \\ \hline
	\end{tabular}
	\caption{Uspješnosti sustava}
	\label{tab:system_performance}
\end{table}
\egroup

Ideje za nadogradnje: počeci i krajevi redaka su obično Ocularu kritični. Tesseract češće prepoznaje interpunkciju kada treba i kada ne treba.

%-------------------------------------------------------------------------------
\chapter{Diskusija}
\label{pog:diskusija}

Komentirati konvergenciju računalnog vida, neuralnih mreža, NLP-a i OCR-a.


%--- ZAKLJUČAK / CONCLUSION ----------------------------------------------------
\chapter{Zaključak}
\label{pog:zakljucak}

\blindtext


%--- LITERATURA / REFERENCES ---------------------------------------------------
% Literatura se automatski generira iz zadane .bib datoteke / References are automatically generated from the supplied .bib file
% Upiši ime BibTeX datoteke bez .bib nastavka / Enter the name of the BibTeX file without .bib extension
\bibliography{library}



%--- SAŽETAK / ABSTRACT --------------------------------------------------------

% Sažetak na hrvatskom
\begin{sazetak}
  Cilj rada nadići je uspješnost gotovih sustava za optičko raspoznavanje teksta na starijim knjigama hrvatskoga jezika koristeći se nenadziranim metodama učenja, uz predobradu i naknadnu obradu. Razmatraju se najznačajniji slobodno dostupni OCR alati prikladni zadatku, Tesseract, OCR sustav opće namjene koji održava Google, te Ocular, razvijen specifično za primjenu na antikvarnim dokumentima. Nakon treniranja i optimiziranja hiperparametara Oculara, uspoređen je s Tesseractom gdje se pokazuje da usprkos starijoj arhitekturi u bitnome nadjačava Tesseract, ali uz određena ograničenja. Konačno, izveden je sustav glasanja kojim se postiže veća uspješnost od one samostalnih modela.
\end{sazetak}

\begin{kljucnerijeci}
  OCR; optičko raspoznavanje teksta; računalni vid; Ocular; Tesseract;
\end{kljucnerijeci}


% Abstract in English
\begin{abstract}
  The aim of the paper is to surpass the accuracy of out-of-the-box systems at Optical Character Recognition of historical documents in the Croatian language relying on unsupervised learning methods, preprocessing and postprocessing. The most appropriate freely available OCR tools are evaluated, namely Tesseract, a general-purpose OCR system maintained by Google, and Ocular, developed specifically for use on historical documents. After training and optimizing Ocular's hyperparameters it is compared to Tesseract where it is shown that despite its older architecture Ocular in the main still bests Tesseract, with certain caveats. Finally, a voting-based system is implemented which achieves greater success than each model alone.
\end{abstract}

\begin{keywords}
  OCR; Optical Character Recogniton; Computer Vision; Ocular; Tesseract;
\end{keywords}


%--- PRIVITCI / APPENDIX -------------------------------------------------------

% Sva poglavlja koja slijede će biti označena slovom i riječi privitak / All following chapters will be denoted with an appendix and a letter
\backmatter

\chapter{The Code}

\Blindtext


\end{document}
